\documentclass{article}
\usepackage{gvv}
\title{\underline{\textbf{jee2022-paper1}}}
\date{}

\begin{document}
\maketitle
\begin{enumerate}
	\item considering only the principal values of the inverse trigonometric functions, the value of
		$\frac{3}{2} \cos^{-1} \sqrt\frac{2}{2+\pi^{2}} + \frac{1}{4} \sin^{-1}\frac{2\sqrt{2}\pi}{2+\pi^{2}} + \tan^{-1}\frac{\sqrt{2}}{\pi}is \underline{\hspace{2cm}}$

	\item Let $\alpha$ be a positive real number. let f: $\mathbb{R} \to \mathbb{R}$ and g: $(\alpha, \infty) \to \mathbb{R}$ be the functions defined by 
		f(x)=$\sin(\frac{\pi x}{12})$ and g(x)=$\frac{2\log_e(\sqrt{x}-\sqrt{\alpha})}{\log_e(e^{\sqrt{x}}-e^{\sqrt{\alpha}})}$
then the value of $\lim_{x \to \alpha}f(g(x))$ is 
\underline{\hspace{2cm}}
	
	\item In a study about a pandemic, data of 900 persons was collected. It was found that\\ 
\begin{itemize}
  \item 190 persons had symptom of fever, 
  \item 220 persons had symptom of cough, 
  \item	220 persons had symptom of breathing problem, 
  \item	330 persons had symptom of fever or cough or both,
  \item	350 persons had symptom of cough or breathing problem or both, 
  \item 340 persons had symptom of fever or breathing problem or both, 
  \item	30 persons had all three symptoms 
\end{itemize}

	If a person is chosen randomly from these 900 persons, then the probability that the person has at 
most one symptom is      \underline{\hspace{2cm}}

	\item Let z be a complex number with non-zero imaginary part.
		If    $\frac{2+3z+4z^2}{2-3z+4z^2}$
		is a real number, then the value of $|z|^2$ is \underline{\hspace{2cm}}

	\item Let $\overline{z}$ denote the complex conjugate of a complex number z and let $i =\sqrt{-1}$ . In the set of complex 
numbers, the number of distinct roots of the equation\\		$\overline{z}-z^2=i(\overline{z}+z^2)$		is \underline{\hspace{2cm}}


	\item Let $l_1,l_2,...,l_{100}$ be consecutive terms of an arithmetic progression with common difference $d_1$, and let $w_1, w_2, ... , w_100$ be consecutive terms of another arithmetic progression with common difference $d_2 $, where $d_1d_2$ = 10. For each i = 1, 2,...,100, let $R_i$ be a rectangle with length $L_i$, width $W_i$ and area $A_i$. If $A_{51}-A_{50}=1000$, then the value of $A_{100}-A_{90}$ is \underline{\hspace{2cm}}
	\item the number of 4-digit integers in the closed interval [2022, 4482] formed by using the digits $0,2,3,4,6,7$ is 
	
		

	\item Let ABC be the triangle with AB = 1, AC = 3 and $\angle BAC=\frac{\pi}{2}$. If a circle of radius $r > 0$ touches the sides AB, AC and also touches internally the circumcircle of the triangle ABC, then the value of r is\underline{\hspace{2cm}}


	\item consider the equation\\
		$\int_1^e\frac{(\log_e x)^\frac{1}{2}}{x(a-(\log_e x)^\frac{3}{2})^2}dx =1$,a $\in (-\infty,0)\cup(1,\infty$) \\
which of the following statements is/are TRUE?

\begin{enumerate}[label=\Alph*.]

		\item No a satisfies the above equation
		\item An integer a satisfies the above equation
		\item An irrational number a satisfies the above equation
		\item More than one a satisfy the above equation
\end{enumerate}                                      

 
	\item Let $a_1, a_2, a_3,...$ be an arithmetic progression with $a_1 = 7$ and common difference 8. Let 
$T_1, T_2, T_3,$... be such that $T_1$ = 3 and $T_{n+1} - t_n = a_n$ for $n\ge1$. Then, which of the following is/are TRUE?\\
\begin{enumerate}[label=\Alph*.]
	\item $T_{20} =1604$  \item$\sum_{K=1}^{20}T_k=10510$
		\item $T_{30}=3454$ \item $\sum_{K=1}^{30}T_k=357610$  
\end{enumerate}


	\item Let $P_1 and P_2$ be two planes given by\begin{itemize}                                         \item $P_1: 10x + 15y + 12z - 60 = 0$,

  \item $P_2 : -2x + 5y + 4z - 20 = 0$.
\end{itemize}
Which of the following straight lines can be an edge of some tetrahedron whose two faces lie on $P_1$
and $P_2$ 
\begin{enumerate}[label=\Alph*.]
	\item $\frac{x-1}{0} =\frac{y-1}{0} =\frac{z-1}{5}$ 
	\item $\frac{x-6}{-5} =\frac{y}{2} =\frac{z}{3}$
	\item $\frac{x}{-2} =\frac{y-4}{5} =\frac{z}{4}$
	\item $\frac{x}{1} =\frac{y-4}{-2} =\frac{z}{3}$
\end{enumerate}

  \item  Let S be the reflection of a point Q with respect to the plane given by\\
\qquad\qquad $\uparrow{r}=-(t+p)\hat{i}+t\hat{j}+(1+p)\hat{k}$\\
 where t, p are real parameters and $\hat{i}, \hat{j} ,\hat{k}$ are the unit vectors along the three positive coordinate 
axes. If the position vectors of Q and S are $10\hat{i}+ 15\hat{j}+ 20\hat{k} $and $\alpha\hat{i}+ \beta\hat{j}+ \gamma\hat{k}$ respectively, then which of the following is/are TRUE ?\\
\begin{enumerate}[label=\Alph*.]
    \item $3(\alpha + \beta) = -101$
    \item $3(\beta + \gamma) = -71$
    \item $3(\gamma + \alpha) = -86$
    \item $3(\alpha + \beta + \gamma) = -121$
\end{enumerate}


	\item Consider the parabola $y^2 = 4x$. Let S be the focus of the parabola. A pair of tangents drawn to the 
parabola from the point p = (-2, 1) meet the parabola at $P_1$ and $P_2$. Let $Q_1$ and $Q_2$ be points on the 
lines $SP_1 and SP_2$ respectively such that $PQ_1$ is perpendicular to $SP_1$ and $PQ_2$ is perpendicular to 
$SP_2$. Then, which of the following is/are TRUE ?

\begin{enumerate}[label=\Alph*.]
	\item $SQ_1=2$
	\item $Q_1Q_2=\frac{3\sqrt{10}}{5}$
	\item $PQ_1=3$
	\item $SQ_2=1$
\end{enumerate}

	\item Let $\mydet{M}$ denote the determinant of a square matrix M. Let g:$[0,\frac{\pi}{2}]\to\mathbb{R}$  be the function defined by\\
		$g(\theta)=\sqrt{f(\theta)-1}+\sqrt{f(\frac{\pi}{2}-\theta)-1}$ where\\
		$f(\theta) = \frac{1}{2} \left| \begin{matrix}
1 & \sin \theta & 1 \\
-\sin \theta & 1 & \sin \theta \\
-1 & -\sin \theta & 1
\end{matrix} \right| + \left| \begin{matrix}
	\sin \pi & \cos \left( \theta + \frac{\pi}{4} \right) & \tan \left( \theta - \frac{\pi}{4} \right) \\
			\sin \left( \theta - \frac{\pi}{4} \right) & - \cos \frac{\pi}{2} & \log_e \left( \frac{4}{\pi} \right) \\
			\cot \left( \theta + \frac{\pi}{4} \right)& \log_e \left( \frac{\pi}{4} \right) & \tan \pi
\end{matrix} \right|$\\
Let p(x) be a quadratic polynomial whose roots are the maximum and minimum values of the function $g(\theta)and p(2) = 2 -\sqrt{2}$. Then, which of the following is/are TRUE ?\\

\begin{enumerate}[label=\Alph*.]
\item $p(\frac{3+\sqrt{2}}{4}) < 0$
\item $p(\frac{1+3\sqrt{2}}{4}) > 0$
\item $p(\frac{5\sqrt{2}-1}{4}) > 0$
\item $p(\frac{5-\sqrt{2}}{4}) < 0$
	\end{enumerate}
	\item consider the following lists:\\
		\centerline{List-I\qquad\qquad\qquad \qquad\qquad\qquad List-II}\\
	$(I){ x \in [ -\frac{2\pi}{3}, \frac{2\pi}{3} ] : \cos x + \sin x = 1 }$ \qquad(P) has two elements\\
	$(II){ x \in [ -\frac{5\pi}{18}, \frac{5\pi}{18} ] : \sqrt{3} \tan 3x = 1 }$\quad \qquad(Q) has three elements \\
        $(III){ x \in [ -\frac{6\pi}{5}, \frac{6\pi}{5} ] : 2 \cos (2x) = \sqrt{3} }$ \qquad(R) has four elements \\
	$(IV){ x \in [ -\frac{7\pi}{4}, \frac{7\pi}{4} ] : \sin x - \cos x = 1 }$ \qquad(s) has five elements\\
	 (V)\qquad\qquad\qquad\qquad\qquad\qquad\qquad\qquad\quad (T) has six elements
	 \mydet
  The correct option is :\\
  \begin{enumerate}[label=\Alph*.]
		  
	  \item  $(I)\to(P);(II)\to(S);(III)\to(P);(IV)\to(S)$
	  \item  $(I)\to(P);(II)\to(P);(III)\to(T);(IV)\to(R)$
	  \item  $(I)\to(Q);(II)\to(P);(III)\to(T);(IV)\to(S)$
	  \item  $(I)\to(Q);(II)\to(S);(III)\to(P);(IV)\to(R)$
\end{enumerate}

	\item Two players, $P_1 and P_2$, play a game against each other. In every round of the game, each player rolls a fair die once, where the six faces of the die have six distinct numbers. Let x and y denote the readings on the die rolled by $P_1 and P_2$ , respectively. If $x > y$, then $P_1$ scores 5 points and $P_2$ 2 scores 0 point. If $x = y$, then each player scores 2 points. If $x < y$, then $P_1$ scores 0 point and $P_2$ scores 5 points. Let $X_i$ and $Y_i$ be the total scores of $P_1 and P_2$, respectively, after playing the $i^{th}$ round.

\begin{center}
\begin{tabular}{l l}
    \textbf{LIST-I} & \textbf{LIST-II} \\
	(I) Probability of $ (X_2 \geq Y_2 )$ is &  $(P) = \frac{3}{8} $\\                                  	(II) Probability of $(X_2 > Y_2 )$ is &\( (Q) = \frac{11}{16} \)\\                                     
	(III) Probability of \( (X_3 = Y_3 )\) is & \((R) = \frac{5}{16} \)\\                              

	(IV) Probability of \( (X_3 > Y_3) \) is & \( (S) = \frac{355}{864} \)\\                                                    
	(V) & $(T)\frac{77}{432}$\\
	\end{tabular}
\end{center}
 the correct option is:
	\begin{enumerate}[label=\Alph*.]
	\item $(I)\to(Q);(II)\to(R);(III)\to(T);(IV)\to(S)$   
	\item$(I)\to(Q);(II)\to(R);(III)\to(T);(IV)\to(S)$
	\item $(I)\to(P);(II)\to(R);(III)\to(Q);(IV)\to(S)$
	\item $(I)\to(P);(II)\to(R);(III)\to(Q);(IV)\to(T)$
	\end{enumerate}

	\item Let p,q amd r  be nonzero real numbers that are the $10^{th}, 100^{th}, and 1000^{th}$ terms of a harmonic progression, respectively. Consider the following system of linear equations:

\[x + y + z = 1\]
\[10x + 100y + 1000z = 0\]
\[qr x + pr y + pq z = 0\]

\begin{center}                                                \begin{tabular}{l l}                                               \textbf{LIST-I} & \textbf{LIST-II} \\
	(I) If \( \frac{q}{r} = 10 \), then the system of linear equations has & (P) \( x = 0,  y = \frac{10}{9}, z = -\frac{1}{9} \) as a solution  \\

	(II) If \( \frac{p}{r} \neq 100 \), then the system of linear equations has & (Q) \( x = \frac{10}{9},  y = -\frac{1}{9},  z = 0 \) as a solution \\

	(III) If \( \frac{p}{q} \neq 10 \), then the system of linear equations has & (R) infinitely many solutions \\                              
	(IV) If \( \frac{p}{q} = 10 \), then the system of linear equations has & (S) no solution \\
	\end{tabular}                                          \end{center}

\begin{enumerate}[label=\Alph*.]		
	\item $(I)\to(T);(II)\to(R);(III)\to(S);(IV)\to(T)$
	\item $(I)\to(Q);(II)\to(S);(III)\to(S);(IV)\to(R)$   
	\item $(I)\to(Q);(II)\to(R);(III)\to(P);(IV)\to(R)$
	\item $(I)\to(T);(II)\to(S);(III)\to(P);(IV)\to(T)$
\end{enumerate}

	\item consider the ellipse $\frac{x^2}{4}+\frac{y^2}{3}=1$ let $H(\alpha,0),0<\alpha<2,$be a point. A straight line drawn through H parallel to the y-axis crosses the ellipse and its auxiliary circle at points E and F respectively, in the first quadrant. The tangent to the ellipse at the point E intersects the positive x-axis at a point G. Suppose the straight line joining F and the origin makes an angle $\phi$ with the positive x-axis

\begin{center}                                        \begin{tabular}{l l}                                       \textbf{LIST-I} & \textbf{LIST-II} \\            (I) if $\varphi=\frac{3}{4}$, then the area of the triangle FGH is & $(p)\frac{(\sqrt{3}-1)^4}{8}$\\        (II) if $\varphi=\frac{\pi}{3}$, then the area of the triangle FGH is & (Q)1\\                              (III) if $\varphi =\frac{\pi}{6}$, then the area of the triangle FGH is & (R)$\frac{3}{4}$\\                (IV) if $\varphi =\frac{\pi}{12}$, then the area of the triangle FGH is & (S)$\frac{1}{2\sqrt{3}}$\\        (V)     & $(T)\frac{3\sqrt3}{2}$\\                    \end{tabular}                                         \end{center}                                                   
\begin{enumerate}[label=\Alph*.]	
\item $(I) \to (R);(II) \to (S);(III) \to (Q);(IV) \to (P)$    
\item $(I) \to (R);(II) \to (T);(III) \to (S);(IV) \to (P)$ 
\item $(I) \to (Q);(II) \to (T);(III) \to (S);(IV) \to (P)$    
\item $(I) \to (Q);(II) \to (S);(III) \to (Q);(IV) \to (P)$\\                                                         
\end{enumerate}

\end{enumerate}
\end{document}
